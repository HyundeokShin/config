/% Copyright 2003--2007 by Till Tantau
% Copyright 2010 by Vedran Mileti\'c
%
% This file may be distributed and/or modified
%
% 1. under the LaTeX Project Public License and/or
% 2. under the GNU Free Documentation License.
%
% See the file doc/licenses/LICENSE for more details.

% $Header: /Users/joseph/Documents/LaTeX/beamer/doc/beamerug-guidelines.tex,v 72b5ec35e666 2015/02/20 21:17:42 joseph $

\section{Guidelines for Creating Presentations}
\label{section-guidelines}

In this section we sketch the guidelines that we try to stick to when we create presentations. These guidelines either arise out of experience, out of common sense, or out of recommendations by other people or books. These rules are certainly not intended as commandments that, if not followed, will result in catastrophe. The central rule of typography also applies to creating presentations: \emph{Every rule can be broken, but no rule may be ignored.}


\subsection{Structuring a Presentation}
\label{section-structure-guidelines}

\subsubsection{Know the Time Constraints}

When you start to create a presentation, the very first thing you should worry about is the amount of time you have for your presentation. Depending on the occasion, this can be anything between 2 minutes and two hours.
\begin{itemize}
\item
  A simple rule for the number of frames is that you should have at most one frame per minute.
\item
  In most situations, you will have less time for your presentation that you would like.
\item
  \emph{Do not try to squeeze more into a presentation than time allows for.} No matter how important some detail seems to you, it is better to leave it out, but get the main message across, than getting neither the main message nor the detail across.
\end{itemize}

In many situations, a quick appraisal of how much time you have will show that you won't be able to mention certain details. Knowing this can save you hours of work on preparing slides that you would have to remove later anyway.

\subsubsection{Global Structure}

To create the ``global structure'' of a presentation, with the time constraints in mind, proceed as follows:
\begin{itemize}
\item
  Make a mental inventory of the things you can reasonably talk about within the time available.
\item
  Categorize the inventory into sections and subsections.
\item
  For very long talks (like a 90 minute lecture), you might also divide your talk into independent parts (like a ``review of the previous lecture part'' and a ``main part'') using the |\part| command. Note that each  part has its own table of contents.
\item
  Do not feel afraid to change the structure later on as you work on the talk.
\end{itemize}

\paragraph{Parts, Section, and Subsections.}

\begin{itemize}
\item
  Do not use more than four sections and not less than two per part.
\end{itemize}

Even four sections are usually too much, unless they follow a very easy pattern. Five and more sections are simply too hard to remember for the audience. After all, when you present the table of contents, the audience will not yet really be able to grasp the importance and relevance of the different sections and will most likely have forgotten them by the time you reach them.
\begin{itemize}
\item
  Ideally, a table of contents should be understandable by itself. In particular, it should be comprehensible \emph{before} someone has heard your talk.
\item
  Keep section and subsection titles self-explaining.
\item
  Both the sections and the subsections should follow a logical pattern.
\item
  Begin with an explanation of what your talk is all about. (Do not assume that everyone knows this. The \emph{Ignorant Audience Law} states: Someone important in the audience always knows less than you think everyone should know, even if you take the Ignorant Audience Law into account.)
\item
  Then explain what you or someone else has found out concerning the subject matter.
\item
  Always conclude your talk with a summary that repeats the main message of the talk in a short and simple way. People pay most attention at the beginning and at the end of talks. The summary is your ``second chance'' to get across a message.
\item
  You can also add an appendix part using the |\appendix| command. Put everything into this part that you do not actually intend to talk about, but that might come in handy when questions are asked.
\item
  Do not use subsubsections, they are evil.
\end{itemize}

\paragraph{Giving an Abstract}

In papers, the abstract gives a short summary of the whole paper in about 100 words. This summary is intend to help readers appraise whether they should read the whole paper or not.
\begin{itemize}
\item
  Since your audience is unlikely to flee after the first slide, in a presentation you usually do not need to present an abstract.
\item
  However, if you can give a nice, succinct statement of your talk, you might wish to include an abstract.
\item
  If you include an abstract, be sure that it is \emph{not} some long text but just a very short message.
\item
  \emph{Never, ever} reuse a paper abstract for a presentation, \emph{except} if the abstract is ``We show $\operatorname{P} = \operatorname{NP}$'' or ``We show $\operatorname{P} \neq \operatorname{NP}$''
\item
  If your abstract is one of the above two, double-check whether your proof is correct.
\end{itemize}

\paragraph{Numbered Theorems and Definitions.}

A common way of globally structuring (math) articles and books is to use consecutively numbered definitions and theorems. Unfortunately, for presentations the situation is a bit more complicated and we would like to discourage using numbered theorems in presentations. The audience has no chance of remembering these numbers. \emph{Never} say things like ``now, by Theorem~2.5 that I showed you earlier, we have \dots'' It would be much better to refer to, say, Kummer's Theorem instead of Theorem~2.5. If Theorem~2.5 is some obscure theorem that does not have its own name (unlike Kummer's Theorem or Main Theorem or Second Main Theorem or Key Lemma), then the audience will have forgotten about it anyway by the time you refer to it again.

In our opinion, the only situation in which numbered theorems make sense in a presentation is in a lecture, in which the students can read lecture notes in parallel to the lecture where the theorems are numbered in exactly the same way.

If you do number theorems and definitions, number everything consecutively. Thus if there are one theorem, one lemma, and one definition, you would have Theorem~1, Lemma~2, and Definition~3. Some people prefer all three to be numbered~1. We would \emph{strongly} like to discourage this. The problem is that this makes it virtually impossible to find anything since Theorem~2 might come after Definition~10 or the other way round. Papers and, worse, books that have a Theorem~1 and a Definition~1 are a pain.
\begin{itemize}
\item
  Do not inflict pain on other people.
\end{itemize}

\paragraph{Bibliographies.}

You may also wish to present a bibliography at the end of your talk, so that people can see what kind of ``further reading'' is possible. When adding a bibliography to a presentation, keep the following in mind:
\begin{itemize}
\item
  It is a bad idea to present a long bibliography in a presentation. Present only very few references. (Naturally, this applies only to the talk itself, not to a possible handout.)
\item
  If you present more references than fit on a single slide you can be almost sure that none of them will be remembered.
\item
  Present references only if they are intended as ``further reading.'' Do not present a list of all things you used like in a paper.
\item
  You should not present a long list of all your other great papers \emph{except} if you are giving an application talk.
\item
  Using the |\cite| commands can be confusing since the audience has little chance of remembering the citations. If you cite the references, always cite them with full author name and year like ``[Tantau, 2003]'' instead of something like ``[2,4]'' or ``[Tan01,NT02]''.
\item
  If you want to be modest, you can abbreviate your name when citing yourself as in ``[Nickelsen and T., 2003]'' or ``[Nickelsen and T, 2003]''. However, this can be confusing for the audience since it is often not immediately clear who exactly ``T.'' might be. We recommend using the full name.
\end{itemize}

\subsubsection{Frame Structure}
\label{section-frame-guidelines}
\label{section-guidelines-local}

Just like your whole presentation, each frame should also be structured. A frame that is solely filled with some long text is very hard to follow. It is your job to structure the contents of each frame such that, ideally, the audience immediately sees which information is important, which information is just a detail, how the presented information is related, and so on.

\paragraph{The Frame Title}

\begin{itemize}
\item
  Put a title on each frame. The title explains the contents of the frame to people who did not follow all details on the slide.
\item
  The title should really \emph{explain} things, not just give a cryptic summary that cannot be understood unless one has understood the whole slide. For example, a title like ``The Poset'' will have everyone puzzled what this slide might be about. Titles like ``Review of the Definition of Partially Ordered Sets (Posets)'' or ``A Partial Ordering on the Columns of the Genotype Matrix'' are \emph{much} more informative.
\item
  Ideally, titles on consecutive frames should ``tell a story'' all by themselves.
\item
  In English, you should \emph{either} \emph{always} capitalize all words in a frame title except for words like ``a'' or ``the'' (as in a title), \emph{or} you \emph{always} use the normal lowercase letters. Do \emph{not} mix this; stick to one rule. The same is true for block titles. For example, do not use titles like ``A short Review of Turing machines.'' Either use ``A Short Review of Turing Machines.'' or ``A short review of Turing machines.'' (Turing is still spelled with a capital letter since it is a name).
\item
  In English, the title of the whole document should be capitalized, regardless of whether you capitalize anything else.
\item
  In German and other languages that have lots of capitalized words, always use the correct upper-/lowercase letters. Never capitalize anything in addition to what is usually capitalized.
\end{itemize}

\paragraph{How Much Can I Put On a Frame?}

\begin{itemize}
\item
  A frame with too little on it is better than a frame with too much on it. A usual frame should have between 20 and 40 words. The maximum should be at about 80 words.
\item
  Do not assume that everyone in the audience is an expert on the subject matter. Even if the people listening to you should be experts, they may last have heard about things you consider obvious several years ago. You should always have the time for a quick reminder of what exactly a ``semantical complexity class'' or an ``$\omega$-complete partial ordering'' is.
\item
  Never put anything on a slide that you are not going to explain during the talk, not even to impress anyone with how complicated your subject matter really is. However, you may explain things that are not on a slide.
\item
  Keep it simple. Typically, your audience will see a slide for less than 50 seconds. They will not have the time to puzzle through long sentences or complicated formulas.
\item
  Lance Fortnow, a professor of computer science, claims: PowerPoint users give better talks. His reason: Since PowerPoint is so bad at typesetting math, they use less math, making their talks easier to understand.

  There is some truth in this in our opinion. The great math-typesetting capabilities of \TeX\ can easily lure you into using many more formulas than is necessary and healthy. For example, instead of writing {\catcode `|=12``Since $\left|\{x \in \{0,1\}^* \mid x \sqsubseteq y\}\right| < \infty$}, we have\dots''\ use ``Since $y$ has only finitely many prefixes, we have\dots''

  You will be surprised how much mathematical text can be reformulated in plain English or can just be omitted. Naturally, if some mathematical argument is what you are actually talking about, as in a math lecture, make use of \TeX's typesetting capabilities to your heart's content.
\end{itemize}

\paragraph{Structuring a Frame}

\begin{itemize}
\item
  Use block environments like |block|, |theorem|, |proof|, |example|, and so on.
\item
  Prefer enumerations and itemize environments over plain text.
\item
  Use |description| when you define several things.
\item
  Do not use more than two levels of ``subitemizing.'' \beamer\ supports three levels, but you should not use that third level. Mostly, you should not even use the second one. Use good graphics instead.
\item
  Do not create endless |itemize| or |enumerate| lists.
\item
  Do not uncover lists piecewise.
\item
  Emphasis is an important part of creating structure. Use |\alert| to highlight important things. This can be a single word or a whole sentence. However, do not overuse highlighting since this will negate the effect.
\item
  Use columns.
\item
  \emph{Never} use footnotes. They needlessly disrupt the flow of reading. Either what is said in the footnote is important and should be put in the normal text; or it is not important and should be omitted (\emph{especially} in a presentation).
\item
  Use |quote| or |quotation| to typeset quoted text.
\item
  Do not use the option |allowframebreaks| except for long bibliographies.
\item
  Do not use long bibliographies.
\end{itemize}

\paragraph{Writing the Text}

\begin{itemize}
\item
  Use short sentences.
\item
  Prefer phrases over complete sentences. For example, instead of ``The figure on the left shows a Turing machine, the figure on the right shows a finite automaton.''\ try ``Left: A Turing machine. Right: A finite automaton.'' Even better, turn this into an itemize or a description.
\item
  Punctuate correctly: no punctuation after phrases, complete punctuation in and after complete sentences.
\item
  \emph{Never} use a smaller font size to ``fit more on a frame.'' \emph{Never ever} use the \emph{evil} option |shrink|.
\item
  Do not hyphenate words. If absolutely necessary, hyphenate words ``by hand,'' using the command~|\-|.
\item
  Break lines ``by hand'' using the command~|\\|. Do not rely on automatic line breaking. Break where there is a logical pause. For example, good breaks in ``the tape alphabet is larger than the input alphabet'' are before ``is'' and before the second ``the.'' Bad breaks are before either ``alphabet'' and before ``larger.''
\item
  Text and numbers in figures should have the \emph{same} size as normal text. Illegible numbers on axes usually ruin a chart and its message.
\end{itemize}

\subsubsection{Interactive Elements}

Ideally, during a presentation you would like to present your slides in a perfectly linear fashion, presumably by pressing the page-down-key once for each slide. However, there are different reasons why you might have to deviate from this linear order:
\begin{itemize}
\item
  Your presentation may contain ``different levels of detail'' that may or may not be skipped or expanded, depending on the audience's reaction.
\item
  You are asked questions and wish to show supplementary slides.
\item
  You present a complicated picture and you have to ``zoom out'' different parts to explain details.
\item
  You are asked questions about an earlier slide, which forces you to find and then jump to that slide.
\end{itemize}

You cannot really prepare against the last kind of questions. In this case, you can use the navigation bars and symbols to find the slide you are interested in, see \ref{section-navigation-bars}.

Concerning the first three kinds of deviations, there are several things you can do to prepare ``planned detours'' or ``planned short cuts''.
\begin{itemize}
\item
  You can add ``skip buttons.'' When such a button is pressed, you jump over a well-defined part of your talk. Skip button have two advantages over just pressing the forward key is rapid succession: first, you immediately end up at the correct position and, second, the button's label can give the audience a visual feedback of what exactly will be skipped. For example, when you press a skip button labeled ``Skip proof'' nobody will start puzzling over what he or she has missed.
\item
  You can add an appendix to your talk. The appendix is kept ``perfectly separated'' from the main talk. Only once you ``enter'' the appendix part (presumably by hyperjumping into it), does the appendix structure become visible. You can put all frames that you do not intend to show during the normal course of your talk, but which you would like to have handy in case someone asks, into this appendix.
\item
  You can add ``goto buttons'' and ``return buttons'' to create detours. Pressing a goto button will jump to a certain part of the presentation where extra details can be shown. In this part, there is a return button present on each slide that will jump back to the place where the goto button was pressed.
\item
  In \beamer, you can use the |\againframe| command to ``continue'' frames that you previously started somewhere, but where certain details have been suppressed. You can use the |\againframe| command at a much later point, for example only in the appendix to show additional slides there.
\item
  In \beamer, you can use the |\framezoom| command to create links to zoomed out parts of a complicated slide.
\end{itemize}

\subsection{Using Graphics}

Graphics often convey concepts or ideas much more efficiently than text: A picture can say more than a thousand words. (Although, sometimes a word can say more than a thousand pictures.)
\begin{itemize}
\item
  Put (at least) one graphic on each slide, whenever possible. Visualizations help an audience enormously.
\item
  Usually, place graphics to the left of the text. (Use the |columns| environment.) In a left-to-right reading culture, we look at the left first.
\item
  Graphics should have the same typographic parameters as the text: Use the same fonts (at the same size) in graphics as in the main text. A small dot in a graphic should have exactly the same size as a small dot in a text. The line width should be the same as the stroke width used in creating the glyphs of the font. For example, an 11pt non-bold Computer Modern font has a stroke width of 0.4pt.
\item
  While bitmap graphics, like photos, can be much more colorful than the rest of the text, vector graphics should follow the same ``color logic'' as the main text (like black~= normal lines, red~= highlighted parts, green~= examples, blue~= structure).
\item
  Like text, you should explain everything that is shown on a graphic. Unexplained details make the audience puzzle whether this was something important that they have missed. Be careful when importing graphics from a paper or some other source. They usually have much more detail than you will be able to explain and should be radically simplified.
\item
  Sometimes the complexity of a graphic is intentional and you are willing to spend much time explaining the graphic in great detail. In this case, you will often run into the problem that fine details of the graphic are hard to discern for the audience. In this case you should use a command like |\framezoom| to create anticipated zoomings of interesting parts of the graphic, see Section~\ref{section-zooming}.
\end{itemize}


\subsection{Using Animations and Transitions}

\begin{itemize}
\item
  Use animations to explain the dynamics of systems, algorithms, etc.
\item
  Do \emph{not} use animations just to attract the attention of your audience. This often distracts attention away from the main topic of the slide. No matter how cute a rotating, flying theorem seems to look and no matter how badly you feel your audience needs some action to keep it happy, most people in the audience will typically feel you are making fun of them.
\item
  Do \emph{not} use distracting special effects like ``dissolving'' slides unless you have a very good reason for using them. If you use them, use them sparsely. They \emph{can} be useful in some situations: For example, you might show a young boy on a slide and might wish to dissolve this slide into a slide showing a grown man instead. In this case, the dissolving  gives the audience visual feedback that the young boy ``slowly becomes'' the man.
\end{itemize}


\subsection{Choosing Appropriate Themes}

\beamer\ comes with a number of different themes. When choosing a theme, keep the following in mind:
\begin{itemize}
\item
  Different themes are appropriate for different occasions. Do not become too attached to a favorite theme; choose a theme according to occasion.
\item
  A longer talk is more likely to require navigational hints than a short one. When you give a 90 minute lecture to students, you should choose a theme that always shows a sidebar with the current topic highlighted so that everyone always knows exactly what's the current ``status'' of your talk is; when you give a ten-minute introductory speech, a table of contents is likely to just seem silly.
\item
  A theme showing the author's name and affiliation is appropriate in situations where the audience is likely not to know you (like during a conference). If everyone knows you, having your name on each slide is just vanity.
\item
  First choose a presentation theme that has a layout that is appropriate for your talk.
\item
  Next you might wish to change the colors by installing a different color theme. This can drastically change the appearance of your presentation. A ``colorful'' theme like |Berkeley| will look much less flashy if you use the color themes |seahorse| and |lily|.
\item
  You might also wish to change the fonts by installing a different font theme.
\end{itemize}


\subsection{Choosing Appropriate Colors}

\begin{itemize}
\item
  Use colors sparsely. The prepared themes are already quite colorful (blue~= structure, red~= alert, green~= example). If you add more colors for things like code, math text, etc., you should have a \emph{very} good reason.
\item
  Be careful when using bright colors on white background, \emph{especially} when using green. What looks good on your monitor may look bad during a presentation due to the different ways monitors, beamers, and printers reproduce colors. Add lots of black to pure colors when you use them on bright backgrounds.
\item
  Maximize contrast. Normal text should be black on white or at least something very dark on something very bright. \emph{Never} do things like ``light green text on not-so-light green background.''
\item
  Background shadings decrease the legibility without increasing the information content. Do not add a background shading just because it ``somehow looks nicer.''
\item
  Inverse video (bright text on dark background) can be a problem during presentations in bright environments since only a small percentage of the presentation area is light up by the beamer. Inverse video is harder to reproduce on printouts and on transparencies.
\end{itemize}


\subsection{Choosing Appropriate Fonts and Font Attributes}

Text and fonts literally surround us constantly. Try to think of the last time when there was no text around you within ten meters. Likely, this has never happened in your life! (Whenever you wear clothing, even a swim suit, there is a lot of text right next to your body.) The history of fonts is nearly as long as the history of civilization itself. There are tens of thousands of fonts available these days, some of which are the product of hundreds of years of optimization.

Choosing the right fonts for a presentation is by no means trivial and wrong choices will either just ``look bad'' or, worse, make the audience have trouble reading your slides. This user's guide cannot replace a good book on typography, but in the present section you'll find several hints that should help you setup fonts for a \beamer\ presentation that look good. A font has numerous attributes like weight, family, or size. All of these have an impact on the usability of the font in presentations. In the following, these attributes are described and advantages and disadvantages of the different choices are sketched.

\subsubsection{Font Size}
\label{section-sizes}

Perhaps the most obvious attribute of a font is its size. Fonts are traditionally measured in ``points.'' How much a point is depends on whom you ask. \TeX\ thinks a point is the 72.27th part of an inch, which is 2.54 cm. On the other hand, PostScript and Adobe think a point is the 72th part of an inch (\TeX\ calls this a big point). There are differences between American and European points. Once it is settled how much a point is, claiming that a text is in ``11pt'' means that the ``height'' of the letters in the font are 11pt. However, this ``height'' stems from the time when letters where still cast in lead and refers to the vertical size of the lead letters. It thus does not need to have any correlation with the actual height of, say, the letter x or even the letter M. The letter x of an 11pt Times from Adobe will have a height that is different from the height of the letter x of an 11pt Times from UTC and the letter x of an 11pt Helvetica from Adobe will have yet another height.

Summing up, the font size has little to do with the actual size of letters. Rather, these days it is a convention that 10pt or 11pt is the size a font should be printed for ``normal reading.'' Fonts are designed so that they can optimally be read at these sizes.

In a presentation the classical font sizes obviously lose their meaning. Nobody could read a projected text if it were actually 11pt. Instead, the projected letters need to be several centimetres high. Thus, it does not really make sense to specify ``font sizes'' for presentations in the usual way. Instead, you should try to think of the number of lines that will fit on a slide if you were to fill the whole slide with line-by-line text (you are never going to do that in practice, though). Depending on how far your audience is removed from the projection and on how large the projection is, between 10 and 20 lines should fit on each slide. The less lines, the more readable your text will be.

In \beamer, the default sizes of the fonts are chosen in a way that makes it difficult to fit ``too much'' onto a slide. Also, it will ensure that your slides are readable even under bad conditions like a large room and only a small projection area. However, you may wish to enlarge or shrink the fonts a bit if you know this to be more appropriate in your presentation environment.

Once the size of the normal text is settled, all other sizes are usually defined relative to that size. For this reason, \LaTeX\ has commands like |\large| or |\small|. The actual size these commands select depends on the size of normal text.

In a presentation, you will want to use a very small font for text in headlines, footlines, or sidebars since the text shown there is not vital and is read at the audience's leisure. Naturally, the text should still be large enough that it actually \emph{can} be read without binoculars. However, in a normal presentation environment the audience will still be able to read even |\tiny| text when necessary.

However, using small fonts can be tricky. Many PostScript fonts are just scaled down when used at small sizes. When a font is used at less than its normal size, the characters should actually be stroked using a slightly thicker ``pen'' than the one resulting from just scaling things. For this reason, high quality multiple master fonts or the Computer Modern fonts use different fonts for small characters and for normal characters. However, when you use a normal Helvetica or Times font, the characters are just scaled down. A similar problem arises when you use a light font on a dark background. Even when printed on paper in high resolution, light-on-dark text tends to be ``overflooded'' by the dark background. When light-on-dark text is rendered in a presentation this effect can be much worse, making the text almost impossible to read.

You can counter both negative effects by using a bold version for small text.

In the other direction, you can use larger text for titles. However, using a larger font does not always have the desired effect. Just because a frame title is printed in large letters does not mean that it is read first. Indeed, have a look at the cover of your favorite magazine. Most likely, the magazine's name is the typeset in the largest font, but your attention will nevertheless first go to the topics advertised on the cover. Likewise, in the table of contents you are likely to first focus on the entries, not on the words ``Table of Contents.'' Most likely, you would not spot a spelling mistake there (a friend of mine actually managed to misspell \emph{his own name} on the cover of his master's thesis and nobody noticed until a year later). In essence, large text at the top of a page signals ``unimportant since I know what to expect.'' So, instead of using a very large frame title, also consider using a normal size frame title that is typeset in bold or in italics.

\subsubsection{Font Families}
\label{section-guidelines-serif}

The other central property of any font is its family. Examples of font families are Times or Helvetica or Futura. As the name suggests, a lot of different fonts can belong to the same family. For example, Times comes in different sizes, there is a bold version of Times, an italics version, and so on. To confuse matters, font families like Times are often just called the ``font Times.''

There are two large classes of font families: serif fonts and sans-serif fonts. A sans-serif font is a font in which the letters do not have serifs (from French \emph{sans}, which means ``without''). Serifs are the little hooks at the ending of the strokes that make up a letter. The font you are currently reading is a serif font. \textsf{By comparison, this text is in a sans-serif font.} Sans-serif fonts are (generally considered to be) easier to read when used in a presentation. In low resolution rendering, serifs decrease the legibility of a font. However, on projectors with very high resolution serif text is just as readable as sans-serif text. A presentation typeset in a serif font creates a more conservative impression, which might be exactly what you wish to create.

Most likely, you'll have a lot of different font families preinstalled on your system. The default font used by \TeX\ (and \beamer) is the Computer Modern font. It  is the original font family designed by Donald Knuth himself for the \TeX\ program. It is a mature font that comes with just about everything you could wish for: extensive mathematical alphabets, outline PostScript versions, real small caps, real oldstyle numbers, specially designed small and large letters, and so on.

However, there are reasons for using font families other than Computer Modern:
\begin{itemize}
\item
  The Computer Modern fonts are a bit boring if you have seen them too often. Using another font (but not Times!) can give a fresh look.
\item
  Other fonts, especially Times and Helvetica, are sometime rendered better since they seem to have better internal hinting.
\item
  The sans-serif version of Computer Modern is not nearly as well-designed as the serif version. Indeed, the sans-serif version is, in essence, the serif version with different design parameters, not an independent design.
\item
  Computer modern needs much more space than more economic fonts like Times (this explains why Times is so popular with people who need to squeeze their great paper into just twelve pages). To be fair, Times was specifically designed to be economic (the newspaper company publishing The Times needed robust, but space-economic font).
\end{itemize}

A small selection of alternatives to Computer Modern:
\begin{itemize}
\item
  Latin Modern is a Computer Modern derivate that provides more characters, so it's not considered a real alternative. It's recommended over Computer Modern, though.
\item
  Helvetica is an often used alternative. However, Helvetica also tends to look boring (since we see it everywhere) and it has a very large x-height (the height of the letter~x in comparison to a letter like~M). A large x-height is usually considered good for languages (like English) that use uppercase letters seldom and not-so-good for languages (like German) that use uppercase letters a lot. (We have never been quite convinced by the argument for this, though.) Be warned: the x-height of Helvetica is so different from the x-height of Times that mixing the two in a single line looks strange. The packages for loading Times and Helvetica provide options for fixing this, though.
\item
  Futura is, in our opinion, a beautiful font that is very well-suited for presentations. Its thick letters make it robust against scaling, inversion, and low contrast. Unfortunately, while it is most likely installed on your system somewhere in some form, getting \TeX\ to work with it is a complicated process. However, it has been made a lot simpler with modern typesetting engines such as |luatex| and |xetex|.
\item
  Times is a possible alternative to Computer Modern. Its main disadvantage is that it is a serif font, which requires a high-resolution projector. Naturally, it also used very often, so we all know it very well.
\item
  DejaVu, a derivate of Bitstream Vera is also a very good and free alternative. TrueType version that comes with OpenOffice.org is complicated to get to work with \TeX, but |arev| \LaTeX\ package provides an easy way to use Type 1 version named Bera. It has both sans-serif and serif versions; |arev| provides both.
\end{itemize}

Families that you should \emph{not} use for normal text include:
\begin{itemize}
\item
  All monospaced fonts (like Courier).
\item
  Script fonts (which look like handwriting). Their stroke width is way too small for a presentation.
\item
  More delicate serif fonts like Stempel and possibly even Garamond (though Garamond is really a beautiful font for books).
\item
  Gothic fonts. Only a small fraction of your audience will be able to read them fluently.
\end{itemize}

There is one popular font that is a bit special: Microsoft's Comic Sans. On the one hand, there is a website lobbying for banning the use of this font. Indeed, the main trouble with the font is that it is not particularly well-readable and that math typeset partly using this font looks terrible. On the other hand, this font \emph{does} create the impression of a slide ``written by hand,'' which gives the presentation a natural look. Think twice before using this font, but do not let yourself be intimidated.

One of the most important rules of typography is that you should use as little fonts as possible in a text. In particular, typographic wisdom dictates that you should not use more than two different families on one page. However, when typesetting mathematical text, it is often necessary and useful to use different font families. For example, it used to be common practice to use Gothic letters to denote vectors. Also, program texts are often typeset in monospace fonts. If your audience is used to a certain font family for a certain type of text, use that family, regardless of what typographic wisdom says.

A common practice in typography is to use a sans-serif fonts for titles and serif fonts for normal text (check your favorite magazine). You can \emph{also} use two different sans-serif fonts or two different serif fonts, but you then have to make sure that the fonts look ``sufficiently different.'' If they look only slightly different, the page will look ``somehow strange,'' but the audience will not be able to tell why. For example, do not mix Arial and Helvetica (they are almost identical) or Computer Modern and Baskerville (they are quite similar). A combination of Gills Sans and Helvetica is dangerous but perhaps possible. A combination like Futura and Optima is certainly OK, at least with respect to the fonts being very different.

\subsubsection{Font Shapes: Italics and Small Capitals}
\label{section-italics}
\label{section-smallcaps}

\LaTeX\ introduces the concept of the \emph{shape} of a font. The only really important ones are italic and small caps. An \emph{italic} font is a font in which the text is slightly slanted to the right \emph{like this}. Things to know about italics:
\begin{itemize}
\item
  Italics are commonly used in novels to express emphasis. However, especially with sans-serif fonts, italics are typically not ``strong enough'' and the emphasis gets lost in a presentation. Using a different color or bold text seems better suited for presentations to create emphasis.
\item
  If you look closely, you will notice that italic text is not only slanted but that different letters are actually used (compare a and \emph{a}, for example). However, this is only true for serif text, not for sans-serif text. Text that is only slanted without using different characters is called ``slanted'' instead of ``italic.'' Sometimes, the word ``oblique'' is also used for slanted, but it sometimes also used for italics, so it is perhaps best to avoid it. Using slanted serif text is very much frowned upon by typographers and is considered ``cheap computer typography.'' However, people who use slanted text in their books include Donald Knuth.

  In a presentation, if you go to the trouble of using a serif font for some part of it, you should also use italics, not slanted text.
\item
  The different characters used for serif italics have changed much less from the original handwritten letters they are based on than normal serif text. For this reason, serif italics creates the impression of handwritten text, which may be desirable to give a presentation a more ``personal touch'' (although you can't get very personal using Times italics, which everyone has seen a thousand times). However, it is harder to read than normal text, so do not use it for text more than a line long.
\end{itemize}

The second font shape supported by \TeX\ are small capital letters. Using them can create a conservative, even formal impression, but some words of caution:
\begin{itemize}
\item
  Small capitals are different from all-uppercase text. A small caps text leaves normal uppercase letters unchanged and uses smaller versions of the uppercase letters for normal typesetting lowercase letters. Thus the word ``German'' is typeset as \textsc{German} using small caps, but as \uppercase{German} using all uppercase letters.
\item
  Small caps either come as ``faked'' small caps or as ``real'' small caps. Faked small caps are created by just scaling down normal uppercase letters. This leads to letters the look too thin. Real small caps are specially designed smaller versions of the uppercase letters that have the same stroke width as normal text.
\item
  Computer Modern fonts and expert version of PostScript fonts come with real small caps (though the small caps of Computer Modern are one point size too large for some unfathomable reason---but your audience is going to pardon this since it will not be noticed anyway). ``Simple'' PostScript fonts like out-of-the-box Helvetica or Times only come with faked small caps.
\item
  Text typeset in small caps is harder to read than normal text. The reason is that we read by seeing the ``shape'' of words. For example, the word ``shape'' is mainly recognized by seing one normal letter, one ascending letter, a normal letter, one descending letter, and a normal letter. One has much more trouble spotting a misspelling like ``shepe''  than ``spape''. Small caps destroy the shape of words since \textsc{shape}, \textsc{shepe} and \textsc{spape} all have the same shape, thus making it much harder to tell them apart. Your audience will read small caps more slowly than normal text. This is, by the way, why legal disclaimers are often written in uppercase letters: not to make them appear more important to you, but to make them much harder to actually read.
\end{itemize}

\subsubsection{Font Weight}

The ``weight'' of a font refers to the thickness of the letters. Usually, fonts come as regular or as bold fonts. There often also exist semibold, ultrabold (or black), thin, or ultrathin (or hair) versions.

In typography, using a bold font to create emphasis, especially within normal text, is frowned upon (bold words in the middle of a normal text are referred to as ``dirt''). For presentations this rule of not using bold text does not really apply. On a presentation slide there is usually very little text and there are numerous elements that try to attract the viewer's attention. Using the traditional italics to create emphasis will often be overlooked. So, using bold text, seems a good alternative in a presentation. However, an even better alternative is using a bright color like red to attract attention.

As pointed out earlier, you should use bold text for small text unless you use an especially robust font like Futura or DejaVu.
